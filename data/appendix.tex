% !TEX encoding = UTF-8
\addcontentsline{toc}{chapter}{附录A~~攻读硕士学位期间发表论文及科研工作情况}%添加到目录中
\chapter*{附录A~~~~攻读硕士学位期间发表论文及科研工作情况}
\appedixfigtabnum{A}%重新计算附图和表的标题号和计数号,参数是附录A,B或者C...
%\setlength{\parindent}{0em}

%\vfill
%\hangafter=1\hangindent=2em\noindent
%\setlength{\parindent}{2em}

附录是正文主体的补充。下列内容作为附录。

\begin{itemize}
	\item 攻读学位期间发表的(含已录用,并有录用通知书的)与学位论文相关的学术论文。必须另页单列论文目录,格式同“参考文献”。
	\item 由于篇幅过大,或取材于复制件不便编入正文的材料、数据。
	\item 对本专业同行有参考价值,但对一般读者不必阅读的材料。
	\item 论文中使用的符号意义、单位缩写、程序全文及有关说明书。
	\item 附件:计算机程序清单、软磁盘、鉴定证书、获奖奖状或专利证书的复印件等。
\end{itemize}

对于一些不宜放入正文中、但作为毕业论文(设计)又是不可缺少的部分,或有重要参考价值的内容,可编入毕业论文(设计)的附录中。例如,过长的公式推导、重复性的数据、图表、程序全文及其说明等。论文的附录依序用大写正体A,B,C……编序号,如:附录A。附录中的图、表、式等另行编序号,与正文分开,也一律用阿拉伯数字编码,但在数码前冠以附录序码,如:图A1;表B2;式(B3)等。