% !Mode:: "TeX:UTF-8"
\chapter{湖南师范大学数学与统计学院}

\section{学院介绍}

数学与统计学院,其前身数学系为1938年国立师范学院成立之初所设立的七个系之一。现设有数学系、信息与计算科学系、统计与金融数学系和数学研究中心、计算研究所、数学奥林匹克研究所。拥有数学与应用数学、信息与计算科学、统计学、数据科学与大数据技术4个本科专业;数学、统计学2个一级学科博士点以及课程与教学论(数学)硕士点、学科教学(数学)专业硕士点和应用统计专业硕士点;数学、统计学2个博士后科研流动站。拥有国家重点(培育)学科—基础数学,国家“211”工程重点建设学科、湖南省国内一流建设学科—数学和湖南省国内一流培育学科—统计学;拥有国家第一类特色专业建设点、湖南省重点专业—数学与应用数学。数学与应用数学专业、信息与计算科学和统计学专业先后入选国家级一流本科专业建设点;拥有“计算与随机数学”教育部重点实验室,“复杂系统的控制与优化”、“应用统计与数据科学”2个湖南省高校重点实验室,湖南省首批高校科技创新团队“数学中的现代分析及应用”,湖南省高校教学团队“数学基础课程”等科学研究和人才培养平台。2020年成为湖南国家应用数学中心核心共建单位。

数学与统计学院现有在编教职工110人,其中教授35人,副教授30人,高级工程师1人,博士生导师28人,硕士生导师69人。享受国务院政府特殊津贴专家3人,国家优秀青年基金获得者2人,教育部新世纪人才计划人选4人,湖南省新世纪121人才工程人选3人,湖南省“芙蓉学者”特聘教授3人,湖南省“百人计划”(青年)专家1人,湖南省普通高校学科带头人6人,湖湘高层次人才聚集工程创新人才2人,湖南省杰出青年基金获得者4人,湖南省湖湘英才入选2人,教育部教师教育师资出国访学研修班成员1人。科学计算导师团队入选湖南省首届优秀研究生导师团队,“统计学及其应用”导师团队入选湖南省研究生优秀教学团队。

学院与美国、英国、法国、德国、芬兰、瑞典、日本、澳大利亚、新加坡、香港等国家和地区的著名院校和科研院所建立了广泛的联系,并联合开展科学研究与人才培养,每年来学院访问和讲学的国内外著名学者达70余人次,学院教师出访达20余人次。

80多年来,学院已为国家输送各类毕业生三万余人,校友遍布海内外。学院现有全日制本科生一千余人、研究生四百余人,已形成多规格、多层次的办学格局。

面向未来,数学与统计学院秉承“仁、爱、精、勤”的校训,按照“突出重点、彰显特色、整体推进、协调发展”的发展思路,朝着“把数学学科建设成为湖南一流、国内先进、国际上有一定影响的学科,提升统计学办学实力”的办学目标奋力迈进!

